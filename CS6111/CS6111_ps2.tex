\documentclass[11pt]{exam}
\printanswers
\usepackage{amsmath,amssymb,complexity}
\usepackage{datetime,enumerate,palatino}
\usepackage{setspace}
\usepackage[utf8]{inputenc}
\usepackage[T1]{fontenc}

\newcommand{\N}{{\mathbb{N}}}
\newcommand{\bool}{\{0,1\}}
\newcommand{\owf}{\mathsf{OWF}}
\newcommand\myeq{\mathrel{\overset{\makebox[0pt]{\mbox{\normalfont\tiny\sffamily def}}}{=}}}
\begin{document}
\setstretch{1.1}

\hrule
\vspace{3mm}
\noindent
{\sf IITM-CS6111 : Foundations of Cryptography  \hfill Assignment \#2 }
\vspace{3mm} \\
\noindent
{\sf Topic: One-way functions \hfill}
\vspace{3mm}
\hrule

\hrule
\vspace{3mm}
\noindent
{\sf Roll No: CS14B043 \hfill Full Name: Harshal Gawai}
\vspace{3mm}
\hrule

{\small 
\begin{itemize}
\itemsep 0pt
\item All answers must be written in mathematically rigorous form.
\item Use LaTeX to generate the pdf file of your solution. \textbf{NO} photographs/scanned copies of hand written solutions are accepted. 
\item Students are allowed to form groups (maximum of $3$ members) for this assignment. However, referring to internet etc is strictly disallowed.
\item Submit solutions at moodle link before the deadline one per team. Clearly mention the names and roll numbers of all the teammates involved.
\end{itemize}
}
% uncomment the lines below and fill name and roll number of teammates here



\begin{questions}

\question[10] 
Prove that there is no one-way function $f\colon\bool^n\to \bool^{\lfloor \log n \rfloor}$.
%\begin{solution}
%    Uncomment the lines above and below and write your solution here
%\end{solution}
\question[10]

Prove or disprove: If $f:\bool^n\to\bool^n$ is a one-way function, then the function $g:\bool^n\to\bool^{n-\lfloor \log n \rfloor}$ which outputs the $n-\lfloor \log n \rfloor$ higher order bits of $f$ is also a one-way function. (i.e $\forall x \in \bool^n$, if $f(x)=b_1b_2\cdots b_n$ then $g(x)=b_1b_2\cdots b_{n-\lfloor \log n \rfloor}$)
%\begin{solution}
%    Uncomment the lines above and below and write your solution here
%\end{solution}

\question[10]({\bf Expanding $\owf$s}) Prove that if $f$ is a one-way function, then the function $g$ defined by $g(x_1,x_2)\myeq\left(f(x_1),x_2\right)$ where $|x_1|=|x_2|$, is also a one-way function. Observe that $g$ reveals half of its input, but is nevertheless one-way.
% \begin{solution}
%     Let's assume \textbf{g} is not one-way function.\\
%     But, since \textbf{g} is computable in polynomial time if f is, so there
%     is a probabilistic polynomial-time algorithm \textbf{\emph A} and a positive polynomial $q(n)$ such that\\
%     \begin{align}
%         Pr[Invert_{(A,g)}(n)=1 \geq 1/q(n)]
%     \end{align}
%     We construct a ppt-algorithm  as follows:\\
%     When  is given $1^{n}$ and y as input, it uniformly chooses x ← $\{0, 1\}^n$ , and then simulates A on input $1^{2n}$ and (x, y).\\
%     The output of  is then simply the second half of the output of A.\\
%     Whenever \textbf{\emph A} successfully inverts g on (x, y), i.e. it outputs (x, x' ) with f (x' ) = y, then  successfully inverts y.\\
%     Hence, the success probability of  is at least that of A, and hence non-negligible.\\
%     Thus, f is not one-way, which is a contradiction.\\
%     Hence, $g$ defined by $g(x_1,x_2)\myeq\left(f(x_1),x_2\right)$ is also one-way function.\\
    
% \end{solution}
\begin{solution}
    Here, $g(x_1,x_2)\myeq\left(f(x_1),x_2\right)$\\
    Where, $g:\{0,1\}^n \rightarrow \{0,1\}^{n+l(n)}$\\
    $f$ is a one-way function. $f:\{0,1\}^n \rightarrow \{0,1\}^{l(n)}$\\
    \\
    Assume $g$ is not a one-way function\\
    $\implies \exists$ an $A$ such that\\
    \begin{align*}
        &\Pr\limits_{x_1\leftarrow\{0,n\}^n, x_2\leftarrow\{0,n\}^n}[A(g(x_1,x_2)) = g^{-1}(g(x_1,x_2))] > \frac{1}{P(n)}\\
        \implies &\Pr\limits_{x_1\leftarrow\{0,n\}^n, x_2\leftarrow\{0,n\}^n} [g(A(f(x_1),x_2)) = g(x_1,x_2)] > \frac{1}{P(n)}\\
        \implies &\Pr\limits_{x_1\leftarrow\{0,n\}^n, x_2\leftarrow\{0,n\}^n}
        [A(f(x_1) , x_2)[2n:n+1], f(A(f(x_1),x_2)[n:1]) = (f(x_1),x_2)] > \frac{1}{p(n)}\\
        \implies &\Pr\limits_{x_1\leftarrow\{0,n\}^n, x_2\leftarrow\{0,n\}^n}
        [f(A(f(x_1),x_2)[2n:n+1]) = f(x_1)] > \frac{1}{P(n)}\\
        \implies &\Pr\limits_{x_1\leftarrow\{0,n\}^n}
        [A'(f(x_1)) = f^{-1}(f(x_1))] > \frac{1}{P(n)}
    \end{align*}
    Here $A'$ takes $f(x_1)$, where $|f(x_2 = l(n))|$ as input calculates $n$ in polynomial time and then, $A'$ uniformly selects an $x_2 \leftarrow \{0,1\}^n$ and outputs $A(f(x_1),x_2)$\\
    Hence, from the previous conclusion, $A'$ inverts $f$ with non-negligible probability.\\
    That means, $f$ is not a one-way\\
    $\implies$ from contradiction\\
    Therefore,  $g$ is one-way function
    
\end{solution}

\question[10]({\bf Composition of $\owf$s}) Let $f$ be a one-way function. Is $g(x)\myeq f(f(x))$ necessarily a one-way function? What about $g'(x)\myeq f(x)||f(f(x))$? Prove your answers.
\begin{solution}\\
    Let $f$ be a one-way function. \\
    (a) Is $g(x)\myeq f(f(x))$ necessarily a one-way function?\\
    (b) What about $g'(x)\myeq f(x)||f(f(x))$?\\
    \textbf{(a)} \\
    \emph{The function g is not necessarily one-way}\\
    Let h be a length-preserving one-way function and define $f(x)$ as follows:\\
    If $x_{n/2+1} ... x_n = 0^{n/2}$, , then $f(x) = 0^{\mid x\mid}$\\
    Else, $f(x) = h(x_1 · · · x_n/2 )0^{n/2}$ .\\
    Clearly, $f$ is length preserving. \\
    We now show that it is one-way.\\
    Intuitively, any adversary inverting $f$ can be used to invert $h$.
    Formally, assume that there exists a PPT adversary $A$ and a polynomial $p$ such that for infinitely many $n’s$
    \begin{align}
        Pr[A(f(U_n)) \in f^{-1}(f(U_n))] \geq \frac{1}{p(n)}
    \end{align}\\
    Then, let $A^{'}$ be an adversary who receives a value $y \in \bool^{n/2}$ and attempts to find a value $x \in h^{−1} (y)$.\\
    The adversary $A^{'}$ just invokes $A$ upon input $y0^{n/2}$ and outputs the first $n/2$ bits of $A$’s output.\\
    We now analyze the success probability of $A^'$ . First, denote $S_n = \{ x \mid x_{n/2+1} · · · x_n = 0^{n/2} \}$ and note that $Pr[U_n \in S_n ] = 2^{−n/2}$ . Next, note that
    \begin{align}
        Pr[A^{'}(h(U_{n/2} )) \in h^{-1}(h(U_{n/2} ))] = Pr[A(f(U_{n} )) \in f^{-1}(f (U_n )) \mid U_n \notin S_n ]
    \end{align}
    Now,\\
    \begin{align*}
        Pr[A(f (U_n )) \in f^{-1} (f (U_n ))] &= Pr[A(f (U_n )) \in f^{-1} (f (U_n )) \mid U_n \in S_n ] . Pr[U_n \in S_n ]\\
        &+ Pr[A(f (U_n )) \in f^{-1} (f (U_n )) \mid U_n \notin S_n ] . Pr[U_n \notin S_n ]\\
        &\leq Pr[U_n \in S_n ] + Pr[A(f (U_n )) \in f^{-1} (f (U_n )) \mid U_n \notin S_n ]\\
        &= Pr[U_n \in S_n ] + Pr[A^{'} (h(U_{n/2} )) \in h^{-1} (h(U_{n/2} ))]\\
        &= \frac{1}{2^{n/2}} + Pr[A^{'} (h(U_{n/2} )) \in h^{-1} (h(U_{n/2} ))]
    \end{align*}
    We therefore have that\\
    \begin{align*}
        Pr[A^{'} (h(U_{n/2} )) \in h^{-1} (h(U_{n/2} ))] &\geq Pr[A(f (U_n )) \in f^{-1} (f (U_n ))] - \frac{1}{2^{n/2}}\\
        &\geq \frac{1}{p(n)} - \frac{1}{2^{n/2}} \geq \frac{1}{q(n)}
    \end{align*}
    for some polynomial $q$. Since $A^{'}$ is PPT, this contradicts the assumption that $h$ is a one-way function. We therefore conclude that $f$ is one-way.\\
    Since, we have established that $f$ is one-way, only thing remaining to show is that $g$ is not one-way.\\
    $\forall x, f (f (x)) = 0^{\mid x\mid}$ (because the last $n/2$ bits of $f (x)$ equal zero, and so $f (f (x))$ equals $0^{\mid x\mid}$ )\\
    Since, it is easy to find a preimage of $0^{\mid x\mid}$ under $g$ (just take $0^{\mid x\mid}$ )\\\\
    Therefore, we have that $g(x) = f (f (x))$ is not one-way.\\
    
    \textbf{(b) Is $g'(x)\myeq f(x)||f(f(x))$? one way function?}\\
    The $g'(x)\myeq f(x)||f(f(x))$ is not necessarily an OWF\\
    Let's prove be counter-example\\
    By intution, it seems that $f^{'}:x \rightarrow f(x)\mid\mid f(f(x))$ is not necessarily a OWF when $f$ is not length-preserving.\\
    Consider the following function $f$:\\
    Let $h$ be an arbitrary length-preserving OWF. $g$ is an arbitrary OWF that compresses its input by a factor two.\\
    $f$ is defined as follows: on input $x$,\\
    
    if $\mid x\mid$ is-odd, then $f(x)=0^{\mid x\mid} \mid\mid h(x)$\\
    else if $\mid x \mid$ is even, then if $\mid x\mid/2$ is also even and $x$ is of the form $x^{'} \mid\mid 0^{3\mid x\mid/4}$, return $x^{'}\mid\mid 0^{\mid x\mid/4}$.\\
    Otherwise, if $x$ is of the form $x^{''}\mid\mid 0^{\mid x\mid/2}$, return $h(x^{''})$. In all other cases, return $g(x)$.\\
    So, $f$ maps n-bit inputs to $n/2$-bit outputs when $n$ is even\\
    And, to 2n-bit outputs when $n$ is odd.\\
    one can prove that $f$ is indeed an OWF by picking an appropriate function $g$ whose image does not intersect with $\{0\}^{n}*\{0,1\}^n$\\
    So that on a random input $x$, the probability that $g(x)$ is of the form $x^{'} \mid\mid 0^{3\mid x\mid/4}$ is very small. (i.e. Easy to invert)\\
    Now, let $n = 2 mod 4$ (n is even, and n/2 is odd).\\
    Then when one receives $y=f(x)$ for a random n-bit input $x$,\\
    it is easy to invert $y \mid\mid f(y)$: with very good probability\\
    Here, $x$ is not of the form $x^{'} \mid\mid 0^{3\mid x\mid/4}$ \\
    hence $f(x)=g(x)$ (which is of odd length n/2). \\
    In this case it holds that $f(y)=0^{n/2} \mid\mid h(y)$. \\
    But this makes inverting $y \mid \mid f(y)$ trivial:\\
    $y \mid \mid f(y) = y \mid \mid (0^{n/2} \mid\mid h(y)) = (y \mid \mid 0^{n/2}) \mid\mid h(y)= f(y\mid\mid 0^{3n/2})\mid\mid f(f(y\mid\mid 0^{3n/2})) $ \\
    Hence one can simply return $y\mid\mid 0^{3n/2}$ to successfully invert $f^{'}:x \rightarrow f(x)\mid\mid f(f(x))$\\
    Therefore, $f^{'}:x \rightarrow f(x)\mid\mid f(f(x))$ is not necessarily a OWF when $f$ is not length-preserving.
\end{solution}

\question[10]({\bf Length preserving $\owf$}) Prove that if there exists a one way function, then there exists a length preserving one-way function.

\emph{Hint: Let $f$ be a one-way function and and let $p(.)$ be a polynomial such that $|f(x)|\leq p(|x|)$. (Justify the existence of $p$.) Define $f'(x)\myeq f(x)||10^{p(|x|)-|f(x)|}$. Further modify $f'$ to get a length-preserving function that remains one-way.}
\begin{solution}
    Let's assume that $g$ is a strong one-way function.\\
    we will now construct a length preserving oneway function $f$.\\
    Since $g$ is one-way, there exist polynomial ${p(.)}$ s.t. $\mid g(x)\mid \leq p(\mid x \mid)$ $\forall x$\\
    Define:\\
    \begin{align*}
    g'(x) = g(x)\mid \mid 10\textsuperscript{$p(\mid x \mid)-\mid g(x) \mid$}
    \end{align*}
    this function always has $\mid g^{'}(x)\mid= p(\mid x\mid)$ and it is trivially strongly one way.\\
    Now, one has to force $\mid x \mid = \mid f(x) \mid$\\
    This one can do by taking $p(\mid x\mid)$ bits as input and taking only $\mid x\mid$ of it into account, i.e\\
    \begin{align*}
        f(x\mid\mid y) = g^{'}(x) for \mid y \mid = p(\mid x \mid) - \mid x \mid + 1
    \end{align*}
    Strong one-wayness of \textbf{f} follows from the strong one-wayness of $g^{'}$.\\\\
    Had \textbf{f} not been strongly one way, we could query a PPT adversary of \textbf{f} for $z = f(y)$ get $x = x_1 \mid\mid . . . \mid\mid x_n$ back and try for each $m \in [n]$ whether $g^{'}(x_1\mid\mid . . . \mid\mid x_m) = z$ \\
    and eventually find a preimage of \textbf{z} under $g'$ in polynomial time with non negligible probability.\\
    Hence the contradiction.\\
    Therefore, if there exists a one way function, then there exists a length preserving one-way function.\\
    
\end{solution}

\end{questions}
\end{document}


