\documentclass[11pt]{exam}
\printanswers
\usepackage{amsmath,amssymb,complexity}
\usepackage{datetime,enumerate,palatino}
\usepackage{setspace}
% Use Chancery Font
\DeclareMathAlphabet{\mathpzc}{OT1}{pzc}{m}{it}

\newcommand{\F}{{\mathbb{F}}}
\newcommand{\Z}{{\mathbb{Z}}}
\newcommand{\N}{\mathbb{N}}
\newcommand{\Q}{\mathbb{Q}}




%%%%%%%%%%%%%%%%%%%%%%%%%%%%%%%%%%%%%%%%%%%%%%%%%%%%%%%%%%%%%%%%%%%%%%%%%%%
\begin{document}
\setstretch{1.1}

\hrule
\vspace{3mm}
\noindent
{\sf IITM-CS6111 : Foundations of Cryptography  \hfill Assignment \#1 }
\vspace{3mm} \\
\noindent
{\sf Topic: Perfect Encryption \hfill}

\vspace{3mm}
\noindent
{\sf Roll No: CS14B043\hfill Full Name: Harshal Gawai} 
\vspace{3mm}
\hrule

\begin{questions}
\question[10] ({\bf \textit{Perfect secrecy}}) 
Suppose a Cryptosystem achieves perfect secrecy for a particular plaintext probability distribution. Prove that perfect secrecy is maintained for any plaintext probability distributions.
\begin{solution}

    \textbf{C} = $\{0,1\}^n$        ...message space - \textbf{M} \\
    \textbf{M} = $\{0,1\}^n$        ...cipher space - \textbf{C} \\
    \textbf{K} = $\{0,1\}^n$        ...key space - \textbf{K} \\

    Let ({\bf \textit{Gen, Enc, Dec}}) be encryption scheme \\
    \textbf{Gen} is key-generation algorithm(probabilistic algo.) that outputs \textbf{key k} choosen to some distribution. \\
    \textbf{Enc} is encryption algorithm that takes key $k \in \textbf{K}$ and $m \in  \textbf{M}$ and outputs ciphertext $c \in \textbf{C}$
    \textbf{Dec} is decryption algorithm that takes ciphertext $c \in \textbf{C}$, key $k \in \textbf{K}$ and outputs message $m \in  \textbf{M}$
    
    Let M and C be plain-text and cipher-text events.\\
    Let $P(M=m_i) = P_i$  , $i \in \textbf{N}$, $i \leq \mid\textbf{M}\mid$ denote probability distribution over \textbf{M}\\
    Let above system achieve perfect secrecy.\\
    $\implies$ M and C are independent\\
    $\implies$ Let $P(C=c\mid M=m) = P(C=c)$ , $\forall \textbf{C}$ and $\forall m \in \textbf{M}$\\
    $\implies$ Let $P(C=c\mid M=m_i) = P(C=c\mid M=m_j) = P(C=c)$ , $\forall c \in \textbf{C}$ and $\forall m_i,m_j \in \textbf{M}$\\
     $\implies$ Let $P(\textbf{Enc}(m_i)=c) = P(\textbf{Enc}(m_j)=c) = P(C=c)$ , $\forall c \in \textbf{C}$ and $\forall m_i,m_j \in \textbf{M}$\\
     
    Consider a different probability distribution on plain-text space.\\
    $P(M=m_i) = P_i'$  , $i \in \textbf{N}$, $i \leq \mid\textbf{M}\mid$ and $i \leq \mid \textbf{M} \mid$\\
    Here,\\
%   \Sigma_{i=1}^k x_i 
    $P(C=c) = \Sigma_{i=1}^{i=\mid \textbf{M}\mid} P(C=c\mid M=m_i).P(M=m_i)$ , $m_i \in \textbf{M}$ , $\forall c \in \textbf{C}$ \\
    $\implies$ $P(C=c) = \Sigma_{i=1}^{i=\mid \textbf{M}\mid} P(Enc(m_i)=c).P(M=m_i)$ , $m_i \in \textbf{M}$ , $\forall c \in \textbf{C}$ \\
    $\implies$ $P(C=c) =  P(Enc(m)=c).\Sigma_{i=1}^{i=\mid \textbf{M}\mid}P(M=m_i)$ , $m_i \in \textbf{M}$ ,$m \in M$ , $\forall c \in \textbf{C}$ \\
    $\implies$  $P(C=c) =  P(Enc(m)=c).(1)$ , $m \in \textbf{M}$  , $\forall c \in \textbf{C}$ \\
    $\implies$ $P(C=c) =  P(Enc(m)=c)$ , $m \in \textbf{M}$  , $\forall c \in \textbf{C}$ \\
    $\implies$ $P(C=c) =  P(C=c\mid M=m)$ , $m \in \textbf{M}$  , $\forall c \in \textbf{C}$ \\
    $\implies$ M and C are independent\\
    
    Hence, cryptosystem common for both achieves perfect secrecy under different(any) probability distribution over plaintext, as it acheves perfect secrecy for single probability distribution.\\
    Hence proved\\
    
\end{solution}

%%%%%%%%%%%%%   2   %%%%%%%%%%%%%%%%%%%%%%%%%%%%%%%%%%%%%%%%%%%%%%%%%%%%%%%%%%%%%%%%%%%%%%%%%%%%%%%%%%%%%%%%%%%%%%%%%%%%%%%%%%%%%%%%%%%%%%%%%%%%%%%%%%%%%%%%%%%%%%%%%%%%%%%%%%%%%%%%


\question[10] ({\bf \textit{Affine cipher}}) 
The \textbf{affine cipher} is a type of monoalphabetic substitution cipher, where each letter in an alphabet is mapped to its numeric equivalent, encrypted using a simple mathematical function, and converted back to a letter.
\begin{parts}
\part \textbf{i)} Prove that Affine cipher achieves perfect secrecy iff keys are used with equal probability 1/312.
\part \textbf{ii)} Compute H(k $\mid$ c) and H(k $\mid$ p,c) for Affine cipher (assume keys are equiprobable and plaintext are equiprobable).
\end{parts}
\begin{solution}
    The affine cipher achieves perfect secrecy if every key is used with equal probability1/312 .\par
    A cryptosystem (P,C,K,E,D), where $\mid$P$\mid$ = $\mid$C$\mid$ = $\mid$K$\mid$ provides perfect secrecy if every key is used with equal probability 1/$\mid$K$\mid$  and for every $x \in P$ and for every $y \in  C$ there is a unique key $k \in K$ such that $ek(x)= y$
\end{solution}

%%%%%%%%%%%%%   3   %%%%%%%%%%%%%%%%%%%%%%%%%%%%%%%%%%%%%%%%%%%%%%%%%%%%%%%%%%%%%%%%%%%%%%%%%%%%%%%%%%%%%%%%%%%%%%%%%%%%%%%%%%%%%%%%%%%%%%%%%%%%%%%%%%%%%%%%%%%%%%%%%%%%%%%%%%%%%%%%


\question[10] ({\bf \textit{Shannon's Theorem}}) 
Prove that by redefining the key space, the key-generation algorithm \textbf{\textit{Gen}} chooses a key uniformly at random from the key space, without changing \textbf{Pr[C=c $\mid$ M=m]} for any m, c.
\begin{solution}
    For ({\bf \textit{Gen, Enc, Dec}}) be encryption scheme:\\
    \textbf{Gen} is a probabilistic algorithm. An algorithm which, during its execution, can make some random choices, to outputs \textbf{key k} choosen over some distribution.\\
    It's an algorithm which, during its execution, can make some random choices, which can be modeled as coin tosses.\\
    Assume that Gen picks key k from key space \textbf{K} with probability p.\\
    Since Gen is randomized, this means that a p-fraction of all the random tapes will lead it to generate k as the key.\\
    If one conceptually redefine the key space to be the set of all the random tapes R, then the probability of creating the particular value k (which should no longer be consider a key) is still p. This is because the number of tapes that leads Gen to generate k is still a p-fraction of all random tapes.\\
    Thus, the probability $P(C=c\mid M=m)$ is solely determined by the distribution of the values k, we see that this conceptual change of key space does not change the value of $P(C=c\mid M=m)$.\\
    However, the selection of keys (which are no longer the values k) is now uniformly distributed over the new key space \textbf{R}.\\
    Hence, Key-generation algorithm \textbf{Gen} chooses a key uniformly at random from key space, without changing $P(C=c\mid M=m)$ for any m,c.\\
\end{solution}

%%%%%%%%%%%%%   4   %%%%%%%%%%%%%%%%%%%%%%%%%%%%%%%%%%%%%%%%%%%%%%%%%%%%%%%%%%%%%%%%%%%%%%%%%%%%%%%%%%%%%%%%%%%%%%%%%%%%%%%%%%%%%%%%%%%%%%%%%%%%%%%%%%%%%%%%%%%%%%%%%%%%%%%%%%%%%%%%

\question[10] ({\bf \textit{Shannon's Theorem}}) 
Prove that by redefining the key space, \textbf{\textit{Enc}} is deterministic without changing \textbf{Pr[C=c $\mid$ M=m]} for any m, c. 
\begin{solution}
\textbf{\textit{Enc}} takes a message $m \in M$ and a key $k \in K$, and is randomised (it
gets a number of bits from some random tape that it uses as input as well).
Instead of implicitly getting the random bits, we make it explicit passing them
as input, by redifining the key space to $K \times \mathcal{R}$ (where
$\mathcal{R}$ is the set of all possible random tapes of the aximal length we
could need):

Thus, \textbf{\textit{Enc}} becomes deterministic, as it has all the randomness it needs in the
new-style key.
\end{solution}
%%%%%%%%%%%%%%%%%%%%%%%%%%%%%%%%%%%%%%%%%%%%%%%%%%%%%%%%%%%%%%%%%%%%%%%%%%%%%%%%%%%%%%%%%%%%%%%%%%%%%%%%%%%%%%%%%%%%%%%%%%%%%%%%%%%%%%%%%%%%%%%%%%%%%%%%%%%%%%%%%%%%%%%%%%%%%%%%%%%%%%%%%%%%%%%%%%%%%%%%%%%%%%%%%%%%%%%%%%%%%%%%%%%%%%%%%%%%%%%%%%%%%%%%%%%%%%%%%%

\question[10] ({\bf \textit{Shannon's Theorem}}) 
Prove or refute: An encryption scheme with message space \textbf{M} is perfectly secret iff for every probability distribution over \textbf{M} and every $c_0, c_1 \in C$  we have \textbf{ Pr[C= \boldmath $c_0$] = Pr[C= $c_1$]}. 

\begin{solution}
  Consider a scheme with $1$ bit of plaintext, $3$ bits of key, and $2$ bits of
  ciphertext. The two bits of ciphertext, $c_0$ and $c_1$, are
  obtained as follows:\par

    $c_0$ = $m_0$ \oplus $k_0$
    \par
    $c_1$ = ($k_2$ \AND $k_1$) \oplus $m_0$ \oplus $k_0$
    
  The possible ciphertexts can be seen in the following table:\par
  \begin{centre}
    \begin{tabular}{r|c|c}
       & $0$ & $1$ \\
      \hline
      $(0,0,0)$ & $(0,0)$ & $(1,1)$ \\
      $(1,0,0)$ & $(1,1)$ & $(0,0)$ \\
      $(0,1,0)$ & $(0,0)$ & $(1,1)$ \\
      $(0,0,1)$ & $(0,0)$ & $(1,1)$ \\
      $(1,1,0)$ & $(1,1)$ & $(0,0)$ \\
      $(0,1,1)$ & $(0,1)$ & $(1,0)$ \\
      $(1,0,1)$ & $(1,1)$ & $(0,0)$ \\
      $(1,1,1)$ & $(1,0)$ & $(0,1)$
    \end{tabular}
  \end{centre}
\par \par
  The scheme is perfectly secure:
  \begin{align*}
    P[M = 0|C = (0,0)] &= \frac{P[M = 0] \cdot \left(P[K = (0,0,0)] + P[K=(0,0,1)] + P[K=(0,1,0)]\right)}{P[C=(0,0)]} \\
                       &= \frac{P[M = 0] \cdot \frac{3}{8}}{\frac{6}{16}} =
    P[M=0] \\[16pt]
    P[M = 0|C = (0,1)] &= \frac{P[M = 0] \cdot P[K = (0,1,1)]}{P[C=(0,1)]}  \\
                       &= \frac{P[M = 0] \cdot \frac{1}{8}}{\frac{2}{16}} =
    P[M=0] \\
    \makebox[\widthof{}]{\vdots} & \\
    etc. &
    \end{align*}
  
    but $\frac{3}{8} = P[C = (0,0)] \neq P[C=(0,1)] = \frac{1}{8}$
\par
  Therefore, \textbf{ Pr[C= \boldmath $c_0$] \neq Pr[C= $c_1$]}
\end{solution}


\end{questions}
\end{document}
